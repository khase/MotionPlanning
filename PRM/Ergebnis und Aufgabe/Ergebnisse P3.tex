\documentclass[a4paper]{scrartcl}
\usepackage[latin1]{inputenc}			% Für Umlaute
\usepackage[T1]{fontenc}			% Für richtige Schrift
\usepackage[ngerman]{babel}			% Für neue deutsche Rechtschreibung (Trennung)
\usepackage{color}
\usepackage{amsmath}
\usepackage{amssymb}
\usepackage{graphicx}
\usepackage{textcomp}				% Fürs Gardzeichen

\usepackage{scrpage2}				% Für Kopf- und Fußzeilen
\pagestyle{scrheadings}				% Für Kopf- und Fußzeilen
\clearscrheadfoot					% Für Kopf- und Fußzeilen

\renewcommand{\labelenumi}{(\alph{enumi})}
\renewcommand{\labelenumii}{(\roman{enumii})}

\begin{document}
\titlehead{Wintersemester 2017/2018\\Motion Planning\\Prof. Horsch, Rudi Scheitler}
\author{David Welsch, Ken Hasenbank}
\title{Praktikum 4 - Probabilistic Roadmap (PRM)}
\maketitle

\section*{Beschreibung der Aufgabe}

Das Ziel dieses Praktikums war es, auf Basis einer PRM einen Weg zwischen einer Start- und einer Zielkonfiguration zu finden. \\

Das Konzept von PRM ist es, eine definierte Anzahl an zufälligen Punkten in den vorher definierten Konfigurationsraum. 

\section*{Beschreibung der L"osung}

Zur L"osung haben wir den vorgegeben Code-Rahmen verwendet. Für unseren Code haben wir die Klasse \textit{Bug0} erstellt, welche von der Klasse \textit{BugAlgorithm} erbt. Dann haben wir die Funktionen \textit{update}, sowie \textit{obstacleInWay} bearbeitet.\\
\begin{itemize}
\item In der Funktion \textit{update (Zeile 25-63)} setzen wir zuerst den Punkt \textit{robotPos} auf die geerbte aktuelle Position. Dann prüfen wir in Zeile 29, ob wir bereits am Ziel angekommen sind. Dies wird geprüft, indem der aktuelle Abstand zwischen dem Ziel und dem Roboter mit einem Mindestwert (ebenfalls geerbt) verglichen wird. Ist dies der Fall, wird der aktuelle Punkt auf die Zielposition gesetzt und die Funktion wird mit dem Return-Wert \textit{true} beendet. Wenn nicht, wird als nächstes mit der Funktion \textit{obstacleInWay} geprüft, ob die Sicht auf das Ziel frei ist. \\
Wenn nein, wird eine boolean-Variable \textit{getroffen} auf \textit{true} gesetzt. Diese bewirkt im weiteren Verlauf, dass ein Counter hochgezählt wird, der den Algorithmus spätestens nach 10.000 Schritten abbricht, um eine Endlosschleife zu vermeiden. \\
Ist kein Hindernis im Weg, wird die Richtung des Roboters wieder in Richtung des Ziels gesetzt. Zudem wird das Hochzählen des Counters wieder gestoppt.\\

In Zeile 60 findet dann die Bewegung des Roboters um einen Schritt statt, während in Zeile 61 die interne Position des Roboters auf den neuen Punkt gesetzt wird.

\item In der Funktion \textit{obstacleInWay (Zeilen 65-86)} ist die Funktionalität implementiert, um das Auftreffen des Roboters auf ein Hindernis zu ermitteln. Zu Beginn der Funktion erstellen wir zuerst eine $3\times3$-Matrix. Diese wird dann so befüllt, dass sie zu einer 90\textdegree -Rotationsmatrix (gegen den Uhrzeigersinn) wird. Anschließend wird in einer for-Schleife für alle vorhandenen Hindernisse zuerst die Distanz zwischen Roboter und Hindernis (Variable \textit{dist}), sowie der Vektor, der auf vom Roboter aus auf den nächsten Punkt des Hindernisses zeigt, ermittelt. Dieser Vektor (Variable \textit{out}) zeigt senkrecht auf das Hindernis. Ist die Distanz kleiner als ein vorgegebener Wert (Konstante \textit{DIST\_MIN}), wird der Vektor \textit{out} mit der Rotationsmatrix multipliziert und normiert. Der resultierende Vektor zeigt parallel entlang des Hindernisses. Anschließend wird die Richtung des Roboters auf diesen neuen Vektor (Variable \textit{out2}) gesetzt. Zudem wird die Nummer des gefundenen Hindernisses per \textit{return}-Befehl zurückgegeben.
\end{itemize}

\section*{Was wir gelernt haben}

Es war sehr interessant, einen solchen Algorithmus zu implementieren, da wir beide noch keinerlei Erfahrungen mit Wegfindungsalgorithmen gesammelt hatten.\\

Wir haben gemerkt, dass wir durch  das Nachvollziehen der Arbeitsweise des Algorithmus diesen sehr gut implementieren konnten. Nach anfänglichen Schwierigkeiten konnten wir den gegebenen Coderahmen gut nutzen.

\end{document}